\chapter{Discussion}
\label{chap:discussion}

\section{Classification performance}
% average precision is only a general metric, the actual precision-recall levell should be configured acording to user reqirenments

% differences in average precision in categories with the similar sample sizes -- due to the challanges in the categoreis themselve, or due to different quality of the dataset

\section{Real-world dataset challenges}
\label{sec:real-world-dataset-challanges}

% how to turn tree of tags into flat structure: propagate labels and use all, use only leafs, hybrid of two? Depends on the desired granularity and available labeled images.

% In this experiment only semi-automatic filtering was performed, where selection of both categories which were filtered as well as labels were made manually while the actually removal of images was automatic. While this is faster, it can't remove all contextual images and some extra images can also be removed. Manual filtering could potentially give even better results.

% labels can be not central on the image. Even if one will apply cutting algorithm, if this object was not presented in the initial set of categories, this could make 

% \section{Configuration recommendations}

\section{Dataset quality vs network architecture}

    % DNN improvement should go together with better dataset
% context improv: some of the issues can be improved automatically, some -- not (portraits)
% GoogleNet

\section{Further work}

\section{Limitations of study}



% NTB dataset
% * context dependent
% * what is on label is not the center of the picture
% * missplaced labels
% * duplicates?
% * tag 'alone' modificator(?)
% * label is the main part of the image or just supplimentary?



% by using one dataset split I can introduce bias (fitting this particular split), but last networks can retrain with different random samples to check this.


% NTB dataset can be restructured and improved in the same way as IMAGENET was made.


% NTB categories were organized in the tree structure with the parent-child connection that could correspond to ``one-of'' or ``part-of'' relationships. Example: sports with football and swimming (one-of); the-body: hand, chest etc.

% (ski-jumping -> ski-flying) -- what kind of relationship?

% Some of the categories were fully contextual: persons -> men (grandfathers, fathers), women (grandmother, mothers), sometimes it is mix: children (boys, girls, sons, daughters).


% would be interesting to add IMAGENET dataset to the NTB one. This would require to merge categories trees


% norwegian specific: stave-churches, prisoners and prison cells can be not distiguashible