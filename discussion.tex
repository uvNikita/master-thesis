\chapter{Discussion}
\label{chap:discussion}

\section{Classification performance}

\section{Real-world dataset challenges}

% \section{Configuration recommendations}

\section{Dataset quality vs network architecture}
% context improv: some of the issues can be improved automatically, some -- not (portraits)
% GoogleNet

\section{Further work}

\section{Limitations of study}



% NTB dataset
% * context dependent
% * what is on label is not the center of the picture
% * missplaced labels
% * duplicates?
% * tag 'alone' modificator(?)
% * label is the main part of the image or just supplimentary?



% by using one dataset split I can introduce bias (fitting this particular split), but last networks can retrain with different random samples to check this.


% NTB dataset can be restructured and improved in the same way as IMAGENET was made.


% NTB categories were organized in the tree structure with the parent-child connection that could correspond to ``one-of'' or ``part-of'' relationships. Example: sports with football and swimming (one-of); the-body: hand, chest etc.

% (ski-jumping -> ski-flying) -- what kind of relationship?

% Some of the categories were fully contextual: persons -> men (grandfathers, fathers), women (grandmother, mothers), sometimes it is mix: children (boys, girls, sons, daughters).


% would be interesting to add IMAGENET dataset to the NTB one. This would require to merge categories trees


% norwegian specific: stave-churches, prisoners and prison cells can be not distiguashible