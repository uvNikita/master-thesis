\chapter{Methods}
\label{chap:methods}

\section{Metadata preparation}
This section describes steps that were made to process NTB images metadata in order to store it in more suitable for this project format.

NTB uses proprietary text format to store images metadata, which is called ``tfo''. These files describe such image information as date, labels, description, photographer's name etc. For the purpose of this research, only image filename, SUBJECT\_HEADING and STORAGE\_TYPE fields were used. SUBJECT\_HEADING contains a list of labels in the Norwegian language, therefore each label was translated into English using dictionary provided by NTB. 15 labels were missing translation and therefore were removed, 85 images were labeled with these tags.

Since all neural networks used in the research were designed to by trained on images with three color channels, grayscale and CMYK images were filtered out from the collection. STORAGE\_TYPE field, which among other information contain indication if the image is grayscale, was used for this purpose -- pictures which had ``svarthvitt'', ``blackandwhite'', ``monochrome'' and ``svartvit'' values in the STORAGE\_TYPE field were removed. However, it was empirically established that not all grayscale images were removed using this method, therefore, in addition, each image was read from the disk and number of its color channels was checked. The first method, therefore was used only for optimization purposes: such images were filtered out without reading its content. From the total number of 966372 processed images, 54045 (5.59\%) were removed due to having one channel (grayscale), 2 were removed due to having four channels (CMYK) and 1 image was removed because it was missing from the image collection on the disk.

As a result of the metadata processing, a Python dictionary was built with image filenames as keys and a corresponding list of labels as values. Since this process was time-consuming (around five to six hours on PC with HDD), the resulting dictionary was saved to disk using pickle \cite{pickle} format and further read and reused in the experiments.

In addition to image metadata, the author was also provided with categories tree definition by NTB. The tree was stored in a text-based format where the parent-child relationship was indicated by TAB character. The tree was transcoded to newick \cite{newick} format by applying recursive function and further used to build in-memory tree representation via Python ETE toolkit \cite{ete3}.


\section{Trial experiments}
    \subsection{General-purpose classification}
    \subsubsection{Selection of categories}
    \subsubsection{Images data preparation}
    \subsubsection{Dataset split}
    % Training/Validation/Test set:
    % * According to Ng it should be around 60\%/20\%/20\%
    % intersection checks
    % the challange in multi-label split
    % "smart" way
    \subsubsection{Training process}
    % caffe framework, pycaffe library, hyper parameters, fine-tuning, caffeNet
    
    % presision = N_correctly_detected_labels / N_total_detected_labels
    % recall = N_correctly_detected_labels / N_all_correct_labels
    
    % update functions (cs231 l.6):
    % * sgd -- old
    % * mu - momentum around 0.5, 0.9 or 0.99
    % * nesterov accelerated momentum (nag) even better (lookahead momentum)
    % * adagrad -- per parameter update
    % * rmsprop -- improved adagrad. Updates do not decay with time to zero
    % * adam -- rmsprop + momentum (beta1 = 0.9, beta2 = 0.995) (default one)
    
    % should decay learning rate (e.g 0.9)
    % epoch -- after seeing all images one time
    \subsubsection{Testing process}
    % transformer, batches, deploy network, metrics
    \subsubsection{Hardware}

\subsection{Specialized classification}
To check potential of specialized classification and LMDB optimizations. Dataset split, training and testing processes, hardware are the same
    \subsubsection{Selection of categories}
    Single-label LMDB is easy therefore skiing category was selected
    \subsubsection{Images data preparation}


\section{Main experiments}
    \subsection{General-purpose classification}
    \subsubsection{Selection of categories}
    
    context-dependent image vs context-dependent term
    
    
    Go trough all categories, remove:
    \begin{itemize}
        \item Context labels: second-hand-cloth, used-cars, counterfeit-money
        \item Vague terms: love, daily-life, future
        \item Combined label: childhood-and-youth-pictures, moos-and-lichen, fires-in-trains (exception: meetings-and-conferences)
        \item Combined term: thriatlon, nordic-combined. Exception: biathlon
        \item Double-meaning: direction (human pointing and signs)
    \end{itemize}
    Decisions were based on the label name only, examples of pictures were investigated only in cases when it was necessary to confirm meaning of the term.
    
    Exceptions: no-people
    Double meaning examples: newsparers (companies and actuall newspapers)
    
    Some pictures were labeled with parents label, in most cases such labels were removed since it is usually not a good idea to mix children (e.g. colors).
    
    
    If selected category (the one that satisfies all criterias) had also children categories which were subcategories of the parent one then pictures labeled with such children categories were also included in the parent category, even if child category by itself was not included in the final selection. For example, buses: school-buses, night-buses, bus-stops (maybe cars example is better). In this example images labeled with school-buses and night-buses (but not bus-stops) will be included in the buses category since there subclass relationship is taking place, even though night-buses doesn't satisfy criterias and thus will not be included in the final selection. This was made to both keep granularity of children categories, and to use information from the parent labels since in some cases parent label contained more pictures than all its descendants combined, more example of such parent categories: flowers, trees and dogs.
    
    examples of relationships:
    \begin{itemize}
        \item Subclass: buses -> school-buses
        \item Part-of: hands -> fingers
    \end{itemize}
    
    YES: dogs: types, NO: fruit: types
    
    non-subclass relationship: tram -> tram-stop
    
    types of categories (maybe according to wordnet):
    \begin{itemize}
        \item actions: swimming, fighting with fire
        \item objects: 
        \item
    \end{itemize}
    
    \subsubsection{Images data preparation}
    LMDB + Multi-labeling
    % two LMDB datasets
    % same dataset for all experiments
    \subsubsection{Dataset split}
    % new method
    \subsubsection{Training process}
    % caffeNet -- same, GoogleNet
    % new batch sizes
    \subsubsection{Testing process}
    % batch sizes, optimized vectorized computations
    \subsubsection{Hardware}

\subsection{Specialized classification}
Image data preparation, dataset split, training and testing processes, hardware are the same.
    \subsubsection{Selection of categories}
    Skiing
    
\section{Libraries and tools}
This section gives description of used libraries and tools in experiments as well as motivation for using them.

\begin{itemize}
    \item \textbf{Caffe} \cite{Jia2014} is a one of the most widely adopted in the industry deep learning framework. In addition to the training framework itself it provides with collection of pretrained models from different research papers \cite{BerkeleyVisionandLearningCenter} and Python language bindings. It also allows to run training on the CUDA \cite{CUDA} enabled graphics cards which significantly improves both training and testing speed \cite{Krizhevsky2012}.
    \item \textbf{NumPy} \cite{numpy} is a Python library for scientific computing. This library was used in the project for optimizations of computations trough vectorized calculations.
    \item \textbf{pandas} \cite{pandas} is a Python library which provides efficient data manipulation trough different datastructures as well as its visualization. It was used for tables representations and drawing charts.
    \item \textbf{Jupyter Notebook} \cite{jupyter} is a web application for live coding, visualization, numerical simulation etc. This application was used for both remote code editing and tables/charts visualizations.
    \item \textbf{ETE Toolkit} \cite{ete3} is a Python framework for tree visualization and analysis. It was used for categories tree parsing (using newick \cite{newick} format), processing and visualization.
    \item \textbf{vagga} \cite{vagga} is a tool that helps to describe and build development environment as well as to run processes inside of it. It also provides different levels of process isolation through Linux namespaces \cite{namespaces} kernel feature. It was used to specify the whole research project environment. Isolation properties that it gives as well as full declarative system specification should improve project reproducability.
\end{itemize}
